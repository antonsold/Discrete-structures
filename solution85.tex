\begin{task}{85}
В группе 20 студентов, из которых 5 математиков-олимпиадников, 10 обычных студентов и 5… кхм… не слишком старательных. Сколькими способами студентов можно разбить на две команды для матбоя, так, чтобы в каждой команде оказался хотя бы один олимпиадник и все не слишком старательные студенты не попали в одну команду?
\end{task}

\begin{solution}
Всего способов набрать студентов в первую команду $\frac{C^{10}_{20}}2$ (вторая команда при этом определится автоматически, разделили на 2, т. к. перестановки команд учтены дважды). Определим число случаев, не удовлетворяющих условию задачи. Во-первых, можно сформировать первую команду так, что в нее попадут все 5 олимпиадников. Тогда способов заполнить оставшиеся места в первой команде $C^{15}_5$ способов. Во-вторых, можно поместить всех раздолбаев в первую команду, тогда оставшиеся места заполняются, опять же, $C^{15}_5$ способами. В предыдущих рассуждениях два раза были учтены случаи, когда все олимпиадники находились в одной команде, а все раздолбаи в другой. Их количество $C^5_{10}$ (выберем из 10 обычных студентов 5 в первую команду). Лишний раз был учтен случай, когда все олимпиадники и раздолбаи оказались в одной команде. Всего получим $\frac{C^{10}_{20}}2 - 2 \cdot C^{15}_5 + C^5_{10} - 1$ способов.
\end{solution}