\begin{task}{11}
На плоскости выбрано конечное количество точек, находящихся в общем положении (никакие три точки не лежат на одной прямой, никакие две не совпадают). Некоторые из выбранных точек соединены отрезками. Если два отрезка пересекаются, то их можно заменить двумя другими с концами в тех же точках (отрезки, имеющие лишь общий конец, не считаются пересекающимися). Может ли этот процесс продолжаться бесконечно? [Необходимо решить задачу непременно методом потенциалов, явно указав, какая функция используется в качестве «потенциала».]
\end{task}

\begin{solution}
Пусть $i$ -- номер итерации ($i \geq 1$). Введем функцию потенциала $f(i)$ -- суммарная длина отрезков. С каждой итерацией суммарная длина отрезков уменьшается (по неравенству треугольника), то есть $f(i+1)<f(i)$. Число различных значений этой функции конечно, так как не превосходит числа графов на $n$ вершинах ($n$ -- число выбранных точек). Значит, функция достигнет своего минимума, то есть такой процесс не может быть бесконечным.
\end{solution}