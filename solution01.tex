\begin{task}{1}
Докажите, что число $\sqrt{2}$ иррационально.
\end{task}

\begin{solution}
Докажем от противного: предположим, что $\sqrt{2}\in\mathbb{Q}$ и придём к противоречию. По предположению, имеет место равенство $\sqrt{2}=\frac{p}{q}$ для некоторых $p,\,q\in\mathbb{N}$. Без ограничения общности будем считать, что $p$ и $q$ не имеют общих простых делителей. Возводя обе части этого равенства в квадрат, получаем \[ 2 = \frac{p^2}{q^2}, \] что эквивалентно равенству $2q^2=p^2$. Поскольку в левой части стоит \textbf{чётное} число, то $p$ обязано быть \textbf{\textit{чётным}}. Значит, $p=2p'$ для некоторого $p'\in\mathbb{N}$. Но тогда $2q^2=4(p')^2$, откуда следует, что $q^2=2(p')^2$, а это влечёт чётность числа $q$. Получаем, что у чисел $p$ и $q$ есть общий простой делитель (а именно, двойка), вопреки нашему предположению. Полученное противоречие завершает доказательство.
\end{solution}