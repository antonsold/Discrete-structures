\begin{task}{96}
Рассмотрим вычеты по модулю $27$. Какие значения может принимать порядок элементов по этому модулю? Для каких элементов по модулю $27$ определено понятие порядка?
\end{task}
\begin{solution}
Порядок по модулю $n$ -- наименьшее целое число $k > 0$, такое, что $a^k \equiv 1 \pmod {n}$
Порядок определен только для взаимно простых с $m$ чисел (в нашем случае, для всех $a$, не кратных $3$). Вычислим функцию Эйлера: $\varphi(27)=27-\frac{27}{3} = 18$. По теореме Эйлера $a^{18} \equiv 1 \pmod {27}$. Тогда порядок числа $a$ является делителем 18 (1, 2, 3, 6, 9, 18). Предположим противное, пусть порядок равен $p < 18$ и $p$ не является делителем 18, то есть $18 = p\cdot c + q$ ($c$ неполное частное от деления $18$ на $p$, остаток от этого деления $q$). Тогда $q < p$. Противоречие.
\end{solution} 