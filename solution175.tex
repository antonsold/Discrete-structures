\begin{task}{175}
Докажите, что матрицу из $\{0, 1\}^{n \times n}$, в каждой строке и столбце которой ровно $k$ единиц можно представить в виде суммы $k$ матриц, в каждой строке и столбце у которых в точности по одной единице.
\end{task}

\begin{solution}
Представим матрицу как матрицу смежности двудольного графа, в которой $A_{ij} = 1$ тогда и только тогда, когда между $i$-й вершиной первой доли и $j$-й вершиной второй есть ребро. Тогда искомое представление равносильно наличию в $k$-регулярном двудольном графе на $n$ вершинах $k$ совершенных паросочетаний, не содержащих общих ребер. Докажем по индукции. Для $k = 1$ утверждение очевидно. Пусть оно выполнено для некоторого $k = p - 1$. Докажем, что оно выполнено для $k = p$. Для этого убедимся, что в $p$-регулярном графе есть совершенное паросочетание. Возьмем произвольное множество $\mathcal{A}$ вершин из первой доли, пусть множество их соседей из второй доли -- множество $\mathcal{B}$. Так как во множестве $\mathcal{A}$ все вершины имеют степень $p$, то общее число ребер в графе $\mathcal{A} \cup \mathcal{B}$ равно $|\mathcal{A}|\cdot p$. С другой стороны, общее число ребер в $\mathcal{A} \cup \mathcal{B}$ не превосходит $|\mathcal{B}|\cdot p$ (так как степень каждой вершины из $\mathcal{B}$ не более $p$). Значит, $|\mathcal{B}| \geq |\mathcal{A}|$. Тогда выполнены условия теоремы Холла, а значит, в графе есть искомое совершенное паросочетание. Удалим из графа ребра, составляющие это паросочетание. Тогда останется $(p - 1)$-регулярный двудольный граф, в котором по предположению индукции существует $p - 1$ совершенных паросочетаний. Значит, в $p$-регулярном графе существует $p$ не пересекающихся по ребрам совершенных паросочетаний. Утверждение индукции доказано.
\end{solution}