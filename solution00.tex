\begin{task}{0}
Пять кроликов рассадили по четырём клеткам. Верно ли, что хотя бы в одной из клеток оказалось по меньшей мере два кролика?
\end{task}

\begin{solution}
Воспользуемся принципом Дирихле. Пусть $A$ --- множество кроликов, которые мы рассаживаем, а $B$ --- множество клеток. По условию, $|A|=5>4=|B|$. По принципу Дирихле, не существует инъективного отображения из $A$ в $B$. Иначе говоря, всегда будет получаться так, что какая-то из клеток вмещает хотя бы двух кроликов.
\end{solution}