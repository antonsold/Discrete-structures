\begin{task}{241}
Дано семейство различных $k$-элементных подмножеств $\mathcal{S} = \{A_1, A_2, \ldots A_m\}$ множества $\{v_1, \ldots, v_n\}$. Назовём элементы $v_i$ и $v_j$ соседями, если они вместе входят хотя бы в одно из множеств $A_k$. Пусть у каждого из элементов $v_j$ существует не более чем $2k$ соседей (включая сам $v_j$). Докажите, что элементы $v1, \ldots v_n$ при всех достаточно больших $k$ можно раскрасить пятью красками, так, чтобы никакое подмножество из $\mathcal{S}$  не было одноцветным.
\end{task}
\begin{solution}
Введем события $B_1, \ldots B_m$ -- множества $A_1, \ldots A_m$ одного цвета. Тогда вероятность каждого из этих событий $5\cdot \left(\frac{1}{5} \right)^k = \left(\frac{1}{5} \right)^{k-1}$. Независимость событий $B_1, \ldots B_m$ будем рассматривать как отсутствие в соответствующих им множествах $A_1, \ldots A_m$ одинаковых элементов. Найдем максимальное число зависимых событий для каждого фиксированного $B_i$. Для каждого из $k$ элементов из $2k$ соседей выберем $k$, сформировав при этом $d = k\cdot \binom{2k}{k}$ множеств, имеющих хотя бы один общий элемент с $B_i$. Докажем теперь существование такого $k$, что \[\mathds{P}(B_i) \leq \frac{1}{(d+1)\cdot e}.\]
Для этого представим биномиальный коэффициент формулой Стирлинга: $\binom{2k}{k} \sim \frac{2^{2k}}{\sqrt{\pi k}} = \frac{4^{k}}{\sqrt{\pi k}}$. Так как степень пятерки растет быстрее степени четверки, то такое $k$ существует. При этом выполнены условия леммы Ловаса, и, значит, $\mathds{P}(\overline{B_1}, \ldots, \overline{B_n}) > 0$, то есть искомая раскраска существует.
\end{solution}