\begin{task}{296}
Пусть $G$ -- простой граф, а $M$ -- паросочетание в нём. Пусть количество рёбер в $M$ равно $m$. Увеличивающим путём в графе $G$ относительно $M$ называется путь (без повторяющихся вершин), в котором рёбра через одно принадлежат $M$, причём первая и последняя вершины пути не инцидентны рёбрам $M$. Докажите, что в $G$ есть паросочетание мощности $(m + k)$ тогда и только тогда, когда в $G$ найдутся $k$ увеличивающих путей без общих вершин.
\end{task}

\begin{solution}
Необходимость. Пусть в графе имеется $k$ увеличивающих путей без общих вершин, и пусть $p \leq k$ из них содержат в сумме $q \leq m$ из $M$. Для каждого из $p$ таких путей возьмем в результирующее паросочетание его нечетные ребра. Так как нечетных ребер на одно больше чем четных в каждом пути, то выбранное таким образом множество ребер имеет мощность $p + q$. Оставшиеся $k - p$ путей -- пути из одного ребра, возьмем их в результирующее паросочетание. Остались $m - q$ ребер из $M$, добавляя их, получим паросочетание мощности $(p + q) + (k - p) + (m - q) = m + k$.\newline
Достаточность. Пусть существует паросочетание мощности $(m + k)$, которое обозначим $M'$. Рассмотрим граф $G = M \cup M'$. Очевидно, все его вершины имеют степень не больше двух. Значит, все его ребра разбиваются на группы.
\begin{enumerate}
    \item 
    Одиночные.
    \begin{enumerate}
        \item 
        Принадлежащие $M$ ($p$ ребер).
        \item
        Принадлежащие $M'$ ($q$ ребер).
    \end{enumerate}
    \item 
    Совпадающие, то есть принадлежащие  $M \cap M'$.
    \item
    Цепочки.
    \begin{enumerate}
        \item Четной длины.
        \item Нечетной длины ($r$ цепочек), начинаются и заканчиваются ребрами из $M$.
        \item Нечетной длины ($s$ цепочек), начинаются и заканчиваются ребрами из $M'$.
    \end{enumerate}
    \item
    Циклы (могут быть только четной длины) ($2v$ ребер).
\end{enumerate}
Удалим из $G$ совпадающие ребра, циклы и цепочки четной длины. Тогда разность мощности множеств $M$ и $M'$ равна $k = q - p + s - r \leq q + s$. Возьмем как увеличивающие пути одиночные ребра, принадлежащие $M'$ и цепочки нечетной длины, с концами в $M'$. Они являются увеличивающими путями без общих вершин, поскольку $M'$ -- паросочетание, а внутри цепочек ребра из $M$ и $M'$ чередуются. Их $q + s \geq k$, значит, утверждение доказано.
\end{solution}