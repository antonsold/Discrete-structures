\begin{task}{201}
Докажите, что среди любых $4^n$ натуральных чисел можно выбрать подмножество из $n$ чисел, каждая пара которых взаимно просты, либо выбрать $n$ чисел, каждая пара которых имеет общий делитель.
\end{task}
\begin{solution}
Пусть в графе вершины соединены ребрами тогда и только тогда, когда числа имеют общий делитель, больший единицы. В задаче требуется оценить сверху число Рамсея $R(n, n)$. Воспользуемся неравенством $R(s,t)\leq C_{s+t-2}^{s-1}$: $R(n,n)\leq C_{2n-2}^{n-1}$. Tак как $2^{2n-2}$ является суммой биномиальных коэффициентов, а $C_{2n-2}^{n-1}$ одним из них, то $R(n,n)\leq C_{2n-2}^{n-1}<2^{2n-2}<4^n$, что и требовалось доказать.
\end{solution}