\begin{task}{197}
Докажите, что число неупорядоченных разбиений числа $n$, в которых ни одно слагаемое не повторяется чаще, чем 2 раза, равно числу неупорядоченных разбиений того же числа на слагаемые, ни одно из которых не делится на 3.
\end{task}
\begin{solution}
Пусть число $s$ разбивается на не делящиеся на 3 слагаемые:
\[n = (a_1 + a_1 + \ldots + a_1) + (a_2 + a_2 + \ldots + a_2) + \ldots + (a_n + a_n + \ldots + a_n) = m_1\cdot a_1 + m_2\cdot a_2 + \ldots + m_n\cdot a_n.\]
Разложим $m_1, m_2, \ldots, m_n$ по степеням тройки, получим:
\[s = \left(\sum _{j = 0}^{\infty}p_{1_j}3^j\cdot a_1 + \sum _{j = 0}^{\infty}p_{2_j}3^j\cdot a_2 + \ldots + \sum _{j = 0}^{\infty}p_{n_j}3^j\cdot a_n \right).\]
Каждый коэффициент $p_{i_j}$ не превосходит двух, все $a_i$ различны, т. к. исходные слагаемые не делились на 3. Значит, существует разбиение на повторяющиеся не более двух раз слагаемые (если $p_{i_j} > 0$, то слагаемое вычислим как $3^j\cdot a_i$, соответственно, если $p_{i_j} = 2$, то таких слагаемых будет два).
\[s = \sum _{i = 1}^{n}\sum _{j = 0}^{\infty} \sum _{k = 1}^{p_{i_j}} 3^j \cdot a_i \eqno(1).\]
Пример:
\begin{multline*}
    17=1+1+1+1+1+1+1+2+2+2+2+2=7\cdot 1+5\cdot 2= \\
    = (2\cdot 3+1)\cdot 1 + (1\cdot 3+2)\cdot 2=3+3+1+6+2+2.
\end{multline*}
Докажем, что это инъекция. Пусть имеются два различных разбиения на не делящиеся на 3 слагаемые. Значит, либо в одно из них входит $a_i$, не входящее в другое, либо множества всех $a_i$ совпадают и разбиения отличаются только коэффициентами при них. В первом случае получившиеся разбиения на слагаемые в сумме (1) будут отличаться тем самым множителем $a_i$. Во втором -- верхним пределом суммирования $p_{i_j}$.

Обратно, пусть существует разбиение на слагаемые $a_1, a_2, \ldots, a_n$, не повторяющиеся более двух раз:
$s = m_1\cdot a_1 + m_2\cdot a_2 + \ldots + m_n\cdot a_n$. Каждое слагаемое представим в виде $p_i \cdot 3^i, i \geq 0$. Сгруппируем сумму по одинаковым $p_i$ и вынесем его за скобки, тогда внутри скобок сумма степеней тройки даст коэффициент $q_i$ перед слагаемым $p_i$. Получим разложение на не делящиеся на 3 слагаемые (т. к. в исходной сумме не было более двух одинаковых слагаемых, то коэффициент при $p_i$ не будет делиться на 3).
\[s = \sum _{i = 1}^{n} \sum_{j = 1}^{q_i} p_i.\]
Пример:
\begin{multline*}
    17=3+3+1+6+2+2=1\cdot 3^1+1\cdot 3^1+1\cdot 3^0 +2\cdot 3^0+2\cdot 3^0+2\cdot 3^0=\\
    =1\cdot (3^1+3^1+3^0)+2\cdot (3^1+3^0+3^0)=1\cdot 7+2\cdot 5=1+1+1+1+1+1+1+2+2+2+2+2.
\end{multline*}
У каждого образа по построенному преобразованию существует прообраз, значит, это сюръекция. \newline
Построена биекция, значит, искомые количества разбиений равны. 
\end{solution}